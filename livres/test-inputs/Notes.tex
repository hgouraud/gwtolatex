
\section{Utilisation de \gwtol{}}

Dans le cadre de mon travail sur la base généalogique chausiaise, j'ai développé,
au fil des ans, un outil en Python me permettant d'éditer périodiquement, et
à la grande satisfaction des Chausiais, une version papier d'un extrait de
cette base.

J'ai récemment (2024) décidé de reprendre ce travail pour en faire un outil
«\,partageable\,», en m'efforçant de rendre aussi générique que possible ce que
j'avais décidé de faire pour mon besoin personnel.

La suite de ce document explique comment utiliser cet outil.

\subsection{Principe général}

\gwtol{} est un outil qui complémente GeneWeb, et présuppose l'existence d'une
base généalogique GeneWeb accessible via le réseau (l'accès en local a amplement
été testé. L'accès en réseau devrait être possible).
\gwtol{} fonctionne en effectuant une succession d'appels vers une base GeneWeb,
et en agglomérant en un seul document PDF (appelé «\,livre\,») les pages résultant
de ces appels. Attention aux paramètres de votre base GeneWeb qui pourraient
conduire GeneWeb à considérer \gwtol{} comme un robot!!
\footnote{TODO mettre une temporisation de 1/2 sec dans \gwtol{}}
Tous les appels vers GeneWeb ne sont pas nécessairement accessibles, en
particulier certains arbres et les tableaux, mais ces restrictions sont
susceptibles d'évoluer si des templates spécific pour TeX sont ajoutés.

Les pages actuellement testées sont les pages personnelles
(p=prénom\&n=patronyme\&occ=occ), les listes de descendances (limitées à
une profondeur de 4; m=D\&v=4), les arbres (m=A et m=D), les graphes
de relations (m=RLM) et les listes par titres (m=TT).

La mise en page de l'ensemble (structuration en chapitres, sections ...,
numérotations diverses, table des matières, index, références croisées)
est réalisée par l'outil \LaTeX{}.

La liste des pages constituant le «\,livre\,» est définie dans un document source
nommé «\,famille.txt\,» conservé dans un répertoire «\,livres\,».
Le document famille.txt peut être agrémenté par des commandes directes au
logiciel \LaTeX{}, et par l'inclusion de fichiers annexes. Ces fichiers annexes
sont conservés dans un sous-répertoire «\,famille-inputs\,» du répertoire livres.

Ainsi, le document que vous lisez a la structure suivante :

\begin{verbatim}
\documentclass[a4paper]{book}

\makeindex

<x Input ./gw2l_dist/Gw2LaTeX-env.tex>

%%%%%%%%%%%%%%%%%%%%
% Edit your own title page data here
%%%%%%%%%%%%%%%%%%%%
\begin{titlepage}
\title{\Huge{Manuel d'utilisation \\
de Gw\kern-0.1em To\LaTeX{}}
\par
\large{Un complément à GeneWeb}}
\author{Henri Gouraud \\
\texttt{Henri.Gouraud@LaPoste.net} \\
\copyright 2024}
\end{titlepage}

% Beginning of document
\begin{document}

%%%%%%%%%%%%%%%%%%%%
% Your document starts here
%%%%%%%%%%%%%%%%%%%%

...
Liste de commandes \gwtol{} et de textes complémentaires
....

%%%%%%%%%%%%%%%%%%
% Standard tail of document
% You may add or suppress Notes pages
%%%%%%%%%%%%%%%%%%
\printindex
% Notes page
\newpage
{\huge {\bf Notes}}
\par
Run mkBook du : {\ddmmyyyydate\today} at \currenttime

GeneWeb commit :
<x Input ./gw2l_dist/gw_version.txt>
\par
GwTo\LaTeX{}{} version :
<x Input ./gw2l_dist/version.txt>
\par
<y \hgbaseversion{}
\par
\pdftexbanner{}
% Additionnal Notes page
\newpage
{\huge {\bf Notes}}
% You may add additional notes pages as needed
\end{document}
\end{verbatim}

\subsection{Liste des commandes \gwtol{}}

Les commandes \gwtol{} présentes dans le document \verb|famille.txt| ont la forme
\verb|<c Commande paramètres>|
Les valeurs de «\,c\,» peuvent être:
\begin{description}[style=nextline]
\item[x] Exécute Commande avec paramètres (voir ci-dessous).
\item[y] Ignore la ligne.
\item[a] Traite la ligne comme une URL vers GeneWeb.
Chaque commande «\,<a\,» provoque l'ouverture d'une section ou sous-section selon
la valeur du paramètre «\,BumpSub\,» ci-dessous. La partie de la ligne comprise entre
«\,>\,» et «\,</a>\,» correspond au titre de la section ou sous-section.
On émet ensuite une commande \LaTeX{} \verb|\index{personne}| où «\,personne\,» est
la personne désignée dans les paramètres de la commande GeneWeb, sauf si le
titre contient lui même une commande \LaTeX{} \verb|\index{texte}|.

Le texte :
\begin{verbatim}
<a href="http://127.0.0.1:2317/%%%BASE%%%?p=leon;n=dupont;oc=0;
    templ=tex;w=%%%PASSWD%%%">Dupont, Léon\label{leondupont}</a>
\end{verbatim}
donne la page \pageref{leondupont} et correspond au premier cas, alors que
\begin{verbatim}
<a href="http://127.0.0.1:2317/%%%BASE%%%?p=lucien;n=dupont;oc=0;
    templ=tex;w=%%%PASSWD%%%">Dupont, Lucien\index{Dupont,
    Lucien dit Le Frère}\label{luciendupont}</a>
\end{verbatim}
correspond au second.
\item[b] Fonctionne comme le cas précédent («\,a\,») sans générer de
nouvelle section. Permet à l'utilisateur de créer lui même la nouvelle
section, et d'ajouter avant le contenu de la page des commentaires spécifiques.

le texte :
\begin{verbatim}
<x Section Dupont, Léon>\label{leondupont}
Commentaire qui n'existe pas dans l'autre cas.
\par
<b href="http://127.0.0.1:2317/%%%BASE%%%?p=leon;n=dupont;oc=0;
    templ=tex;w=%%%PASSWD%%%"></a>
\end{verbatim}
est équivalent (au commentaire près) à 
\begin{verbatim}
<a href="http://127.0.0.1:2317/%%%BASE%%%?p=lucien;n=dupont;oc=0;
    templ=tex;w=%%%PASSWD%%%">Dupont, Léon\label{leondupont}</a>
\end{verbatim}
\end{description}
Toutes les autres lignes du fichier \verb|famille.txt| sont recopiées tel-quel.

Les commande \gwtol{} sont les suivantes :

\begin{description}[style=nextline]
\item[Arbres/Trees] If «\,on\,», mets ImgWidth et PortraitWidth à 1.5.
Si «\,off\,» redonne à ces parametres leurs valeurs par defaut.
\item[Chapter/Section/SubSection/SubSubSection] Nouveau chapitre/section/...
\gwtol{} a besoin de connaitre la numérotation des chapitres et sections afin
de numéroter correctement les images. Il ne faut donc pas utiliser directement
les commandes \LaTeX{} équivalentes (\verb|\chapter|, \verb|\section|...).
\item[CollectImages] «\,on/off\,», Rassembler les images citées dans les notes
d'une personne pour les imprimer à la fin de la page personnelle.
\item[Expand] Cherche à étendre la largeur des cellules en utilisant
les cellules vides adjacentes. Cette commande prend un paramètre indiquant
le nombre de fois où cette opération est effectuée. ATTENTION : cette opération
peut introduire des décalages dans les lignes horizontales des arbres.
\item[FontSize] Définit la taille de la police de caractères\label{fontsize}
(small, tiny, off returns to default).
\item[HighLight] Surligner le nom de la personne désignée par
«\,Patronyme prénom (occ)\,». Omettre «\,(0)\,» si l'occurrence est 0.
\item[Hoffset] Décale la page à droite/gauche. Ne s'applique que pendant les arbres.
\item[Hrule] Ajouter une ligne horizontale de la largeur du texte à la fin
des pages produites par «\,<a\,» ou «\,<b\,».
\item[Incr] Afin de numéroter correctement les images, \gwtol{} a besoin de
connaître les numéros de chapitre, section, etc. Si une commande
\verb|\chapter{xxx}|, \verb|\section{xxx}|, ... est insérée directement
dans le document, il convient d'en informer \gwtol{} en incrémentant
le niveau de section approprié. Cette commande remet à zéro les sections de
niveau supérieur. On regardera l'exemple des fiches individuelles
page \pageref{fiches}. Sans cette commande, les images de la section
{\bf Dupont, Léon} auraient été numérotées 3.1.x.
\item[ImageLabel] Précise le nombre d'éléments pris en compte dans
la numérotation des images (ch, sec, ssec, img-nbr).
\item[ImgWidth] Définit la largeur des images (float).
\item[Input] Inclure dans le livre le fichier désigné. Dans ce fichier,
remplacer les occurrences de \verb|%%%LIVRES%%%|, \verb|%%%BASE%%%| et
\verb|%%%PASSWD%%%| par leurs valeurs telles que définies au lancement du programme.
\item[LaTeX] Insérer dans le livre une commande \LaTeX{}.
\item[NbImgPerLine] Définit le nombre d'images par ligne (defaut 3) et en
déduit la largeur.
\item[Newpage] Passer à la page suivante.
\item[Offset] Si «\,off\,», annule le décalage horizontal et vertical de la page.
Su une valeur numérique est proposée, elle correspond à un décalage horizontal;
\item[PortraitWidth] Définit la largeur des portraits (float).
\item[Print] Imprimer la valeur du paramètre.
\item[Reset] remet à zéro les numéros de chapitre, section, etc.
voir la commande «\,Incr\,» ci-dessus.
\item[Sideways] Imprime la page en mode paysage (utile pour les arbres
qui peuvent être larges).\label{sideways}
\item[TreeMode] 0 imprime l'arbre comme une liste de liste de cellules
(debug, devrait fonctionner pour tous les arbres), 1 imprime l'arbre.
\item[TwoPages] Éclate un arbre sur deux pages (WIP).
\item[Unit] Définit l'unité (cm, mm, pt, em).
\item[Version] Imprime la version du logiciel \gwtol{}.
\item[Voffset] Décale la page à haut/bas. Idem
\item[VignWidth] Définit la largeur des vignettes (float).
\item[WideImages] «\,on/off\,». Donne aux images qui suivent la largeur du texte
(\verb|\textwidth|).

\end{description}

\subsection{Les pages GeneWeb}

GeneWeb produit ses résultats à partir de fichiers templates qui peuvent être
adaptés aux besoins des utilisateurs.
\gwtol{} utilise un jeu de templates conservés dans le dossier tex qui
produisent un mélange de code \LaTeX{} et de pseudo-code.
Les pages GeneWeb qui ont été testées abondamment sont les suivantes :

\begin{description}[style=nextline]
\item[Liste de descendance] Affiche la liste de descendance jusqu'à 4
générations au maximum. Pour faciliter la lecture des descendances plus riches,
la fonction HighLight ci-dessus permet de surligner les personnes
qui font l'objet d'un arbre détaillé (le traditionnel «\,qui suit\,» des généalogies).
Les lignes :
\begin{verbatim}
<x Chapter Listes de descendants>
\label{descendance}
<x HighLight Dupont, Léon>
<a href="http://127.0.0.1:2317/%%%BASE%%%?m=D;p=leon+françois;n=dupont;v=4;t=L;
   spouses=on;parents=on;templ=tex;w=hg:%%%PASSWD%%%">Famille Dupont</a>Dupont (Famille)
<a href="http://127.0.0.1:2317/%%%BASE%%%?m=D;p=leon;n=dupont;v=4;t=L;
   spouses=on;parents=on;templ=tex;w=hg:%%%PASSWD%%%">Famille Léon Dupont</a>Dupont, Léon (Famille)
<x HighLight off>
\end{verbatim}
donnent le résultat de la page \pageref{descendance}.

\item[Page perso] Affiche les prénoms complets et alias, puis
un portrait si disponible, juxtaposé avec les
données d'état-civil, naissance/baptême/décès/inhumation et profession
et le couple de parents si connus.

Ces deux pavés sont suivis par la ou les familles avec la liste des enfants.
Dans ces listes, l'utilisation des puces indique la présence ($\bullet$)
ou l'absence ($\circ$) de descendance (ou d'ascendance pour les parents).
À noter que ces descendances ou ascendances sont celles mémorisées dans la base
et ne reflètent pas strictement la réalité (les parents par exemple ont toujours
une ascendance dans la réalité!!).

Vient ensuite la note éventuelle apportant des commentaires sur la personne,
complétée par une liste des photos sur laquelle la personne apparaît.
On reviendra plus loin (page \pageref{images}) sur cette fonctionnalité.
Le dernier pavé cite les sources d'information pour les données collectées.

\item[Liste par titre] Obtenues avec la commande
\verb|m=TT;sm=S;t=titre;p=lieu;|.

\item[Arbres] Obtenues avec les commandes \verb|m=A;t=T| et \verb|m=D;t=T|.
La mise en page automatique de ces arbres est un problème difficile et
nécessite des tâtonnements an ajustant la profondeur, l'orientation «\,portrait\,»
ou «\,paysage\,» de la page (commande «\,Sideways\,» page \pageref{sideways}) et la
taille de la police de caractères (commande «\,FontSize\,» page \pageref{fontsize}).
Les paramètres «\,Hoffset\,» et «\,Voffset\,» permettent de décaler la page
horizontalement ou verticalement. Le paramètre «\,Expand\,» cherche à élargir
les cellules en occupant le cellules voisines vides.

\item[Pages liées] GeneWeb permet au gestionnaire de la base généalogique
de rassembler des informations dans des documents séparés appelés «\,pages liées\,».
Ces pages sont «\,appelées\,» par la commande GeneWeb \verb|m=NOTES;f=fichier| ou
\verb|[[[fichier/étiquette]]]|. Le contenu de ces pages est traité de la même
manière que les notes individuelles et n'influence pas la numérotation
des sections.

À titre d'exemple, la fiche de
{\bf Léon Dupont}, page \pageref{leondupont} appelle la page lié
\verb|page-liee-test| cachée sous l'étiquette {\bf projet de barrage}
page \pageref{pageliee} dont le contenu (extrait du format archive
base-test.gw) est :

\begin{verbatim}
# extended page "page-liee-test" used by:
#  - person "Léon.0 Dupont"
page-ext page-liee-test
  TITLE=Page liée test
  
  <p align="center"><b>PAGE TEST</b></p>
  
  Contenu d'une page test
  <img SRC="%sm=IM;s=image-test-4.jpg">
  
  Le Diplodocus.<br>
  Dans les années 60, l'ingénieur des mines Albert Caquot<span
  style="display:none" mode="tex">\index[Caquot, Albert]</span>
  imagine un gigantesque projet de barrage sur l'ensemble de la baie du
  Mont Saint Michel utilisant l'énergie marée-motrice. Le Diplodocus participe
  aux travaux de sondage de ce projet.<br>
  (<small>source photo [[Léon/Dupont]]</small>).
end page-ext
\end{verbatim}

\item[Autres pages] \gwtol{} interprète les balises HTML et s'efforce de
les traduire dans un équivalent \LaTeX{}. Dans la mesure où HTML et \LaTeX{}
n'ont pas les mêmes conventions et modèle de mise en page, une traduction
littérale n'est pas possible, et échoue dans les cas un peu complexes,
les tableaux en particulier.

\end{description}

\subsection{Index}

L'utilisation d'un outil automatique permet de construire systématiquement
l'index des personnes citées à un titre ou à un autre dans le livre.
\gwtol{} collecte toutes les références aux personnes
détectées dans les URLs vers ces personnes. Pour les femmes mariées
\gwtol{} produit deux entrées supplémentaires :
\begin{verbatim}
\index{Non de jeune fille, Prénon (ep Patronyme du mari)}
\index{Patronyme du mari, Prénon (née Non de jeune fille)}
\end{verbatim}
et ajoute aussi pour tout le monde les lignes :
\begin{verbatim}
\index{Alias, voir Patronyme, Prénom}
\end{verbatim}
si un alias est défini dans la base.

Au-delà de cette collecte, \gwtol{} permet d'insérer dans le corps des notes
des commandes \LaTeX{} qui produiront une entrée dans l'index sans perturber
l'affichage sur le Web :
\begin{verbatim}
<span style="display:none\,» mode="tex">\index[Aaaa, Bbbb]</span>
\end{verbatim}
permettant d'ajouter dans l'index des personnes citées dans les notes
sans qu'elles soient présentes dans la base.
Le contenu du <span> peut être en fait n'importe quelle commande LaTeX de votre choix.
ATTENTION : pour une raison obscure, il est nécessaire d'encadrer le
contenu des commandes \LaTeX{} par des [ ] plutôt que les { } attendus.

\subsection{Images et index des personnes dans les images}

L'une des forces de GeneWeb est de permettre l'insertion d'images ou de
liens vers des images dans le corps des notes. La version papier d'une base
se doit de reproduire ces images.

Les images appelées avec la commande \verb|m=IM| ont vocation à apparaître
directement dans la partie note de la page. Dans \gwtol{}, ces images seront
affichées directement dans la page avec la largeur définie par
le paramètre «\,ImgWidth\,».

Les images désignées par une URL telle que :
\verb|<a href="%sm=IMH;s=fichier.jpg">étiquette</a>| n'apparaissent pas
directement dans la page, et apparaîtront lorsque l'utilisateur cliquera
sur le lien «\,étiquette\,». Afin de les rendre visibles sur la
version «\,papier\,» de la base, ces images sont rassemblées à la fin
de la page à raison de «\,NbImgPerLine\,» images par ligne (défaut 3).
À l'endroit où elles sont appelées, une étiquette en indice rassemblant
chapitre, section, et numéro d'ordre de l'image est affiché. Cette même 
étiquette apparaître au pieds de l'image.
Voir la page \pageref{leondupont} pour un exemple.

Les vignettes, identifiées par la présence du mot «\,vignette\,» dans
le nom de fichier sont affichées avec la largeur définie par le paramètre
«\,VignWidth\,». Elles apparaissent en général dans une combinaison
\verb|<a href/<img| telle que : \\
\verb|<a href="%sm=IMH;s=fichier.jpg"><img SRC="%sm=IMH;s=fichier-vignette.jpg"></a>|
Voir la page \pageref{leondupont} pour un exemple.

Sans avoir recours à la reconnaissance des visages, \gwtol{} se propose de
rajouter dans l'index la référence des photos où les personnes référencées
dans l'index apparaissent. Pour ce faire, \gwtol{} s'appuie sur le fichier
«\,famille-inputs/who-is-where.txt\,» que l'utilisateur aura préparé.
La structure de ce fichier est la suivante :

\begin{verbatim}
# /1 représente l'occurrence;
# /z indique une personne qui n'est pas présente dans la base
Image-id;Page;Description;nom-de-fichier.jpg
\index{Wicart, Monseigneur}/z
\index{Gardais, Abbé}/1
\index{Marie, Alain}

#Image suivante
\end{verbatim}
\begin{description}[style=nextline]
\item[Image-id] Identifiant unique de l'image
\item[Page] 0 si l'image apparait dans les notes de la base, n si elle apparait
dans l'annexe (voir ci-dessous)
\item[Description] Description de l'image
\item[Nom-de-fichier] Nom du fichier dans le répertoire des images de la base
(\tt{bases/mabase/src/images})
\end{description}

L'index de ce document page~\pageref{index} donne un exemple de ce traitement.

\subsection{Annexe}
\label{annexe}
Quand il n'est pas possible d'incorporer toutes les images souhaitées dans les
notes d'une base, il reste la possibilité d'adjoindre au document produit par
\gwtol{} une annexe qui sera automatiquement ajoutée au fichier .PDF produit.
Cette annexe a sa propre numérotation de pages.

\subsection{La totalité du fichier test.txt}

Voir \verb|\gwtol{}/livres/test.txt|

\subsection{Améliorations}

Idées d'améliorations :
\begin{description}
\item Simplifier, clarifier homogénéiser les commandes.
\item Ajouter les images du carrousel.
\item Test guillemets : "classiques", « classiques », «\,classiques\,»
\end{description}

\subsection{Installation, test et exécution}

\gwtol{} est installé à partir de son répertoire GitHub :
\verb|git clone https://github.com/hgouraud/\gwtol{}|
suivi par
\verb|ocaml configure.ml|, \verb|make distrib| or \verb|make install|

Ces deux dernières commandes construisent un répertoire \verb|gw2l_dist| qui
contient tous les éléments nécessaires au fonctionnement de \gwtol{}.

\verb|./setup-test.sh| organise les données de la configuration de test et
lance l'exécution du test (\verb|./run-test.sh|) qui construit le présent
fichier (dans \verb|gwtolatex/livres|). Cette dernière commande est
suffisante si vous souhaitez relancer le test après une modification
dans les données de test.

À noter que \verb|./setup-test.sh| fait l'hypothèse que le répertoire
\verb|gwtolatex| est «\,à côté\,» du répertoire \verb|geneweb| et relance
l'exécution de \verb|gwd| avec les paramètres appropriés. Il est à noter
aussi qu'il convient d'avoir exécuté \verb|make install| dans le repo de GeneWeb.

\gwtol{} doit être exécuté à partir du répertoire des bases de
GeneWeb\footnote{Cette contrainte, est liée à la sécurité de Geneweb}.
Il convient donc de recopier le dossier \verb|gw2l_dist| dans le
dossier contenant vos bases. Une fois cette copie faite, l'exécution se
fait avec :
\begin{verbatim}
cd dossier_des_bases
./gw2l_dist/mkBook -base ma_base -livres dossier_livres -family famille
\end{verbatim}

Le répertoire \verb|dossier_livres| contient le fichier \verb|famille.txt|
et un dossier
\verb|famille-input| contenant \verb|who-is-where.txt| (optionnel),
le fichier (optionnel) \verb|Annexes.pdf|
et les fichiers éventuels appelés avec la commande \verb|<x Input fichier>|.
Dans ces fichiers, les chaines de caractères \verb|%%%LIVRES%%%|,
\verb|%%%BASE%%%| et \verb|%%%PASSWD%%%| seront remplacées par les valeurs
des paramètres \verb|-livres|, \verb|-base| et \verb|-passwd|.

ATTENTION : la première étape du processus de fabrication du «\,livre\,»
construit une nouvelle version de la base utilisée dans laquelle les
données de référence croisées des images ont été ajoutées à l'aide du
fichier \verb|who-is-where.txt|.
Cette création échouera si des divergences de noms (majuscules, accents)
existent entre la base d'origine et ce fichier.

\subsubsection{Paramètres de lancement}

Le logiciel \verb|mkBook| est lancé avec les paramètres suivants :
\begin{description}
\item[-base] nom de la base utilisée.
\item[-bases] «\,.\,»; mkBook est exécuté dans le dossier bases (WIP).
\item[-gw] dossier des binaires GeneWeb.
\item[-passwd] mot de passe magicien (\verb|user:pwd|) si nécessaire.
\item[-livres] dossier des «\,livres\,».
\item[-family] nom du «\,livre\,».
\item[-famille] idem.
\item[-o] nom du fichier résultat. Par défaut \verb|livres/family.pdf|.
\item[-debug] niveau de debug. Par défaut 0.
\item[-once] run pdflatex once only. Par défaut «\,faux\,».
\item[-v] exécuter pdflatex en mode verbose. Par défaut «\,faux\,».
\item[-help] liste des paramètres.
\end{description}

